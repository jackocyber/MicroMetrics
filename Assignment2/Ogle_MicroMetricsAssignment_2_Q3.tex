% Options for packages loaded elsewhere
\PassOptionsToPackage{unicode}{hyperref}
\PassOptionsToPackage{hyphens}{url}
%
\documentclass[
  12pt,
  landscape]{article}
\usepackage{lmodern}
\usepackage{amssymb,amsmath}
\usepackage{ifxetex,ifluatex}
\ifnum 0\ifxetex 1\fi\ifluatex 1\fi=0 % if pdftex
  \usepackage[T1]{fontenc}
  \usepackage[utf8]{inputenc}
  \usepackage{textcomp} % provide euro and other symbols
\else % if luatex or xetex
  \usepackage{unicode-math}
  \defaultfontfeatures{Scale=MatchLowercase}
  \defaultfontfeatures[\rmfamily]{Ligatures=TeX,Scale=1}
\fi
% Use upquote if available, for straight quotes in verbatim environments
\IfFileExists{upquote.sty}{\usepackage{upquote}}{}
\IfFileExists{microtype.sty}{% use microtype if available
  \usepackage[]{microtype}
  \UseMicrotypeSet[protrusion]{basicmath} % disable protrusion for tt fonts
}{}
\makeatletter
\@ifundefined{KOMAClassName}{% if non-KOMA class
  \IfFileExists{parskip.sty}{%
    \usepackage{parskip}
  }{% else
    \setlength{\parindent}{0pt}
    \setlength{\parskip}{6pt plus 2pt minus 1pt}}
}{% if KOMA class
  \KOMAoptions{parskip=half}}
\makeatother
\usepackage{xcolor}
\IfFileExists{xurl.sty}{\usepackage{xurl}}{} % add URL line breaks if available
\IfFileExists{bookmark.sty}{\usepackage{bookmark}}{\usepackage{hyperref}}
\hypersetup{
  pdftitle={ECON 21110 - Applied Microeconometrics - Assignment 2},
  pdfauthor={Jack Ogle},
  hidelinks,
  pdfcreator={LaTeX via pandoc}}
\urlstyle{same} % disable monospaced font for URLs
\usepackage[margin=1in]{geometry}
\usepackage{graphicx,grffile}
\makeatletter
\def\maxwidth{\ifdim\Gin@nat@width>\linewidth\linewidth\else\Gin@nat@width\fi}
\def\maxheight{\ifdim\Gin@nat@height>\textheight\textheight\else\Gin@nat@height\fi}
\makeatother
% Scale images if necessary, so that they will not overflow the page
% margins by default, and it is still possible to overwrite the defaults
% using explicit options in \includegraphics[width, height, ...]{}
\setkeys{Gin}{width=\maxwidth,height=\maxheight,keepaspectratio}
% Set default figure placement to htbp
\makeatletter
\def\fps@figure{htbp}
\makeatother
\setlength{\emergencystretch}{3em} % prevent overfull lines
\providecommand{\tightlist}{%
  \setlength{\itemsep}{0pt}\setlength{\parskip}{0pt}}
\setcounter{secnumdepth}{-\maxdimen} % remove section numbering
\usepackage{dcolumn}

\title{ECON 21110 - Applied Microeconometrics - Assignment 2}
\author{Jack Ogle}
\date{}

\begin{document}
\maketitle

Problem 3 (a) The causal question is related to proving the Solow Growth
Model through modern data about GDP and GDP growth taken from 1960-2021.
Specifically, we want to study if real income is higher in countries
with higher savings rates and lower in countries with higher values of
\({ n + g + \delta}\). We want to see how log(I/GDP) and
log(\({ n + g + \delta}\)) (X) effect log(GDP) per working age person in
2021 (Y). We assume that g + delta are constant over every country and
equal to 0.05.

\begin{enumerate}
\def\labelenumi{(\alph{enumi})}
\setcounter{enumi}{1}
\item
  The ideal experiment would need a large random sample of countries all
  with the exact same economies and we could measure how Investment /
  GDP and n the average rate of growth of the working age population
  (age defined as 14 to 64) on the log gdp of a working age person in
  2021.
\item
  Some issues with the ideal experiment are that cannot just clone
  countries economies, population, and size. Countries are huge entities
  which have complex and intricate differences that cannot be
  duplicated.
\item
  I might add controls like measures for Human capital to control the
  variability among countries. An example might be the varying levels of
  education. This would increase the credibility of MLR.4 being true
  because educational achievement is correlated to investment and GDP an
  independent variable. Thus if we include it, remove it from our error
  term it will make the Zero Conditional Mean assumption more credible
  and make the assumption that our estimator is unbiased.
\end{enumerate}

\end{document}
