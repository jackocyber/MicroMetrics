% Options for packages loaded elsewhere
\PassOptionsToPackage{unicode}{hyperref}
\PassOptionsToPackage{hyphens}{url}
%
\documentclass[
  12pt,
  landscape]{article}
\usepackage{lmodern}
\usepackage{amssymb,amsmath}
\usepackage{ifxetex,ifluatex}
\ifnum 0\ifxetex 1\fi\ifluatex 1\fi=0 % if pdftex
  \usepackage[T1]{fontenc}
  \usepackage[utf8]{inputenc}
  \usepackage{textcomp} % provide euro and other symbols
\else % if luatex or xetex
  \usepackage{unicode-math}
  \defaultfontfeatures{Scale=MatchLowercase}
  \defaultfontfeatures[\rmfamily]{Ligatures=TeX,Scale=1}
\fi
% Use upquote if available, for straight quotes in verbatim environments
\IfFileExists{upquote.sty}{\usepackage{upquote}}{}
\IfFileExists{microtype.sty}{% use microtype if available
  \usepackage[]{microtype}
  \UseMicrotypeSet[protrusion]{basicmath} % disable protrusion for tt fonts
}{}
\makeatletter
\@ifundefined{KOMAClassName}{% if non-KOMA class
  \IfFileExists{parskip.sty}{%
    \usepackage{parskip}
  }{% else
    \setlength{\parindent}{0pt}
    \setlength{\parskip}{6pt plus 2pt minus 1pt}}
}{% if KOMA class
  \KOMAoptions{parskip=half}}
\makeatother
\usepackage{xcolor}
\IfFileExists{xurl.sty}{\usepackage{xurl}}{} % add URL line breaks if available
\IfFileExists{bookmark.sty}{\usepackage{bookmark}}{\usepackage{hyperref}}
\hypersetup{
  pdftitle={Assignment 5 Question 2},
  pdfauthor={Jack Ogle in collaboration with Eva Haque, Matt Lohrs, and Jack Knickrehm},
  hidelinks,
  pdfcreator={LaTeX via pandoc}}
\urlstyle{same} % disable monospaced font for URLs
\usepackage[margin=1in]{geometry}
\usepackage{graphicx,grffile}
\makeatletter
\def\maxwidth{\ifdim\Gin@nat@width>\linewidth\linewidth\else\Gin@nat@width\fi}
\def\maxheight{\ifdim\Gin@nat@height>\textheight\textheight\else\Gin@nat@height\fi}
\makeatother
% Scale images if necessary, so that they will not overflow the page
% margins by default, and it is still possible to overwrite the defaults
% using explicit options in \includegraphics[width, height, ...]{}
\setkeys{Gin}{width=\maxwidth,height=\maxheight,keepaspectratio}
% Set default figure placement to htbp
\makeatletter
\def\fps@figure{htbp}
\makeatother
\setlength{\emergencystretch}{3em} % prevent overfull lines
\providecommand{\tightlist}{%
  \setlength{\itemsep}{0pt}\setlength{\parskip}{0pt}}
\setcounter{secnumdepth}{-\maxdimen} % remove section numbering
\usepackage{dcolumn}
\usepackage{float}

\title{Assignment 5 Question 2}
\author{Jack Ogle in collaboration with Eva Haque, Matt Lohrs, and Jack
Knickrehm}
\date{}

\begin{document}
\maketitle

\begin{enumerate}
\def\labelenumi{\roman{enumi})}
\tightlist
\item
  Estimating college enrollment.
\item
  Estimate the impact of unionization on business survival, employment,
  output, productivity, and wages. 2
\item
  Estimating housing prices as a measure for willingness to pay for
  clean air.
\end{enumerate}

\begin{enumerate}
\def\labelenumi{\alph{enumi})}
\setcounter{enumi}{1}
\tightlist
\item
  What is the running variable, X?
\item
  PSAT score is the running variable. Because it is continuous and can
  be assumed to be random at the cutoff it works well as a running
  variable for RD.
\end{enumerate}

\begin{enumerate}
\def\labelenumi{\roman{enumi})}
\setcounter{enumi}{1}
\tightlist
\item
  Voting percentage is the running variable, and because voting
  percentage is continuous and can be assumed to be random at the cutoff
  of 51\% it works well as a running variable for RD.
\item
  Total Suspended Particles(TSP's) are the running variable. Because
  amount of pollution in the air is a continuous variable which is
  quasirandom around the cutoff it works well as a RD variable.
\item
  What is the treatment variable, D, and how is it determined by X? Is
  this a sharp or a fuzzy RDD?
\item
  Financial aid is the treatment variable. Because students above the
  cutoff on the PSAT score are more likely to get financial aid, and
  financial aid is likely to encourage students to go to college, it is
  a treatment variable. This is a sharp RDD because there is a well
  defined cutoff which students cannot influence.
\item
  Unionization is the treatment variable. Because companies above the
  cutoff of 51\% are unionized and unions lead to higher bargaining
  power of workers, unionization is chosen as the treatment variable.
  This is a sharp RDD because there is a cutoff that union workers
  cannot influence.
\item
  Federally regulated pollution measure are the treatment variable.
  Because counties above the cutoff of TSP are regulated according to
  their emissions which impacts housing prices, it is the treatment
  variable. This is a fuzzy RDD because this variable can be influenced,
  hence the need to use the IV.
\item
  Argue why (or why not) an RDD has more credible identifying
  assumptions than a multiple linear regression model to answer the
  causal question(
\end{enumerate}

\begin{enumerate}
\def\labelenumi{\alph{enumi})}
\setcounter{enumi}{18}
\tightlist
\item
  of interest?
\item
  Because financial aid is such an important factor in college
  enrollment, we want to measure its specific effect. Effects should be
  continuous as students are similar and the unobservables should be
  continuously related to PSAT score. Therefore because we expect such a
  difference to be attributable to this cutoff we want to use RDD
  instead of SLR. 3
\end{enumerate}

\begin{enumerate}
\def\labelenumi{\roman{enumi})}
\setcounter{enumi}{1}
\tightlist
\item
  Because unionization has this strong cutoff and other effects should
  be continuous between different companies, we should use RDD instead
  of SLR.
\item
  Total Suspended Particles(TSP's) are the running variable. Because
  amount of pollution in the air is a continuous variable which is
  quasirandom around the cutoff it works well as a RD variable.
\end{enumerate}

\end{document}
