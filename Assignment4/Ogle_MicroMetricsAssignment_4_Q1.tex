% Options for packages loaded elsewhere
\PassOptionsToPackage{unicode}{hyperref}
\PassOptionsToPackage{hyphens}{url}
%
\documentclass[
  12pt,
  landscape]{article}
\usepackage{lmodern}
\usepackage{amssymb,amsmath}
\usepackage{ifxetex,ifluatex}
\ifnum 0\ifxetex 1\fi\ifluatex 1\fi=0 % if pdftex
  \usepackage[T1]{fontenc}
  \usepackage[utf8]{inputenc}
  \usepackage{textcomp} % provide euro and other symbols
\else % if luatex or xetex
  \usepackage{unicode-math}
  \defaultfontfeatures{Scale=MatchLowercase}
  \defaultfontfeatures[\rmfamily]{Ligatures=TeX,Scale=1}
\fi
% Use upquote if available, for straight quotes in verbatim environments
\IfFileExists{upquote.sty}{\usepackage{upquote}}{}
\IfFileExists{microtype.sty}{% use microtype if available
  \usepackage[]{microtype}
  \UseMicrotypeSet[protrusion]{basicmath} % disable protrusion for tt fonts
}{}
\makeatletter
\@ifundefined{KOMAClassName}{% if non-KOMA class
  \IfFileExists{parskip.sty}{%
    \usepackage{parskip}
  }{% else
    \setlength{\parindent}{0pt}
    \setlength{\parskip}{6pt plus 2pt minus 1pt}}
}{% if KOMA class
  \KOMAoptions{parskip=half}}
\makeatother
\usepackage{xcolor}
\IfFileExists{xurl.sty}{\usepackage{xurl}}{} % add URL line breaks if available
\IfFileExists{bookmark.sty}{\usepackage{bookmark}}{\usepackage{hyperref}}
\hypersetup{
  pdftitle={ECON 21110 - Applied Microeconometrics - Assignment 4},
  pdfauthor={Jack Ogle in collaboration with Eva Haque, Matt Lohrs, and Jack Knickrehm},
  hidelinks,
  pdfcreator={LaTeX via pandoc}}
\urlstyle{same} % disable monospaced font for URLs
\usepackage[margin=1in]{geometry}
\usepackage{graphicx,grffile}
\makeatletter
\def\maxwidth{\ifdim\Gin@nat@width>\linewidth\linewidth\else\Gin@nat@width\fi}
\def\maxheight{\ifdim\Gin@nat@height>\textheight\textheight\else\Gin@nat@height\fi}
\makeatother
% Scale images if necessary, so that they will not overflow the page
% margins by default, and it is still possible to overwrite the defaults
% using explicit options in \includegraphics[width, height, ...]{}
\setkeys{Gin}{width=\maxwidth,height=\maxheight,keepaspectratio}
% Set default figure placement to htbp
\makeatletter
\def\fps@figure{htbp}
\makeatother
\setlength{\emergencystretch}{3em} % prevent overfull lines
\providecommand{\tightlist}{%
  \setlength{\itemsep}{0pt}\setlength{\parskip}{0pt}}
\setcounter{secnumdepth}{-\maxdimen} % remove section numbering
\usepackage{dcolumn}
\usepackage{float}

\title{ECON 21110 - Applied Microeconometrics - Assignment 4}
\author{Jack Ogle in collaboration with Eva Haque, Matt Lohrs, and Jack
Knickrehm}
\date{}

\begin{document}
\maketitle

Problem 1

\begin{enumerate}
\def\labelenumi{(\alph{enumi})}
\tightlist
\item
  Population Model (1) \[
  JQ_i = \beta_0 +\beta_1econ_i +\beta_2drinking_i +u_i
  \]
\end{enumerate}

\(\beta_0\) is the estimator measuring the average baseline of JQ
indexes. Holding all other variables constant including holding heavy
drinking and being enrolled in applied econometric constant at 0
\(\beta_0\) is the average job quality index JQ.

\(\beta_1\) is the estimator measuring the average effect of taking
applied econometrics on Job quality index holding heavy drinking
constant. On average taking econometrics increases Job Quality index by
\(\beta_1\) standard deviations.

\(\beta_2\) is the estimator measuring the average effect of heavy
drinking on Job quality index holding taking applied econometrics
constant. On average heavy alcohol consumption decreases or increases
Job Quality index by \(\beta_2\) standard deviations.

\begin{enumerate}
\def\labelenumi{(\alph{enumi})}
\setcounter{enumi}{1}
\item
\end{enumerate}

No I do not expect drinking to be truthfully reported by college
students because even if the consumption is anonymously reported
students still might feel guilty or ashamed of how much they drink. They
might think that the researcher would like to hear a certain answer and
in order to be helpful report that specific answer. For these reasons I
believe that they might under report the amount of alcohol they consume.
I also believe that the amount people report drinking is correlated with
measurement error. Heavy drinkers are more prone to under report
compared to people who seldom drink or don't drink at all. The
conditions of an survey that would greatly reduce if not eliminate
measurement error would be if student honestly reported drinking habits.
Additionally if they surveyed honest students who are part of a
randomized sample, such that there was not measurement error or omitted
variable bias.

\begin{enumerate}
\def\labelenumi{(\alph{enumi})}
\setcounter{enumi}{2}
\item
\end{enumerate}

A violation of MLR.4 (the zero conditional mean) would make the the OLS
estimator \(\hat{\beta_1}\) of \(\beta_1\) (1) biased. I would expect
MLR.4 to be Omitted Variable biased here. I expect this because there
are many other factors that are in the error term that are correlated to
the other independent variables. For example, the Economics department
requires all Economic specialization in data science majors to take
Applications of Microeconometrics or a similar alternative in a pool of
classes. Or maybe one of the students is in a fraternity or sorority and
they are required to participate in heavy drinking. So the binary
indicator variables of being a Economics specialization Data Science
Majors and Being a member of a fraternity or sorority are variables in
the error term that would violate MLR.4 because they are correlated to
independent variables.

\begin{enumerate}
\def\labelenumi{(\alph{enumi})}
\setcounter{enumi}{3}
\item
\end{enumerate}

In order for econ\_cost to be a valid instrument for econ if it is
correlated with econ and uncorrelated with the error term. \[
Cov(JQ_i, econcost_i) = Cov(\beta_1econ_i+U_i,econcost_i)
\] \[
Cov(U_i,econcost_i) = 0
\] \[
\beta_1 = \frac{Cov(JQ_i, econcost_i)}{Cov(econ_i, econcost_i)}
\] A consistent estimate \(\hat{\beta_{IV_1}}\) of \(\beta_1\) is

\[
\hat{\beta_{IV_1}} = \frac{\Sigma(JQ_i-\bar{JQ})(econcost_i-\bar{econcost})}{\Sigma(econ_i-\bar{econ})(econcost_i-\bar{econcost})}
\] I believe that the variable econ\_cost or above econcost is not a
valid IV. We are not given any data here so I am going off of the
follwoing assumptions. The econ variable is an indicator variable
indicating a 1 if a student takes econometrics. The econ\_cost variable
randomly assigns students to colleges where the econometrics is a
cheeper class compared to other colleges offering the same class.
Econ\_cost is not a good IV because it is weakly correlated or not
correlated with taking applied econometrics. Students do not care
whether the price of the course is cheaper than the same course offered
at a different college. Transferring colleges is an too expensive an
implicit cost to pay for taking a less explicitly expensive class. The
fact that they are in a college with a less expensive econometrics class
compared to other colleges does not increase nor decrease the
probability that students take the econometric class. Because this IV is
not correlated with econ it is not valid.

\begin{enumerate}
\def\labelenumi{(\alph{enumi})}
\setcounter{enumi}{4}
\item
\end{enumerate}

First-Stage regression: Isolate the part of econ that is correlated with
the error term regress econ on econ\_cost using OLS

\[
econ_i = \pi econcost_i + v 
\] \[
\hat{\pi} = \frac{\hat{Cov}(econcost_i, econ_i)}{\hat{Var}(econcost_i)}
\]

then use \(\hat{\pi}\) to predict the value of econ:
\(\hat{econ_i} = \hat{\pi} econcost_i\)

From here we can calculate the heteroskedasticity-robust F-statistic to
determine if it is relevant.

\begin{enumerate}
\def\labelenumi{(\alph{enumi})}
\setcounter{enumi}{5}
\item
\end{enumerate}

The main problem with the endogenous variable is that the variable is
correlated with the error term resulting in Omitted Variable Bias and
MLR.4 is violated. Our estimator \(\tilde{\beta_1}\) solves the
endogeneity bias problem if econ\_cost is a valid estimator because
MLR.4 is more credibly satisfied. In using a valid instrumental variable
we have successfully taken a variable out of the error term and included
it our regression. We must include the variance of the estimator that is
uncorrelated with the error term. if econ\_cost is a valid IV for econ
we can use econ cost to estimate econ. Then we can get the variance of
econ uncorrelated with the error term. This allows us to run an unbiased
regression because there is overall less correlation between the error
term and the independent variables.

\begin{enumerate}
\def\labelenumi{(\alph{enumi})}
\setcounter{enumi}{6}
\item
\end{enumerate}

We cannot test exogeneity. If econ\_cost is not truly exogenous in the
structural model (1) then it would be endogenous and correlated with the
error term. This would violate MLR.4 and the estimator would be
inconsistent and biased.

\begin{enumerate}
\def\labelenumi{(\alph{enumi})}
\setcounter{enumi}{7}
\item
\end{enumerate}

This Cov(Z, U) = 0 must hold for all IV. However, it is impossible to
observe the error term therefore we must employ economic theory to
intuit an answer. From the response to part d I believe that econ\_cost
is not a valid instrument for econ. This is because the cost of an
econometrics course relative to other econometrics courses at different
colleges is weakly correlated if correlated at all with a students
choice to take that class. If econ is not truly exogenous to the model
then there is omitted variable bias and the variable is correlated to
the error term. Meaning that there is(are) a variable(s) that is(are)
correlated with econ and JQ, but is not included in the regression (eg.
Ability, IQ, KWW). Thus the econ\_cost variable is unlikely to be
exogenous. The IV is not valid as it is not corrolated with econ nor is
it exogenous.

\[
\frac{Cov(econ, U)}{Var(econ)}< \frac{Cov(econcost, U)}{Cov(econcost,econ)} => \lim_{n\to\infty}{plim\hat{\beta_1}} < \lim_{n\to\infty}{plim\tilde{\beta_1}}
\] This shows that the estimator is more consistent without the IV and
we should stick with the original estimator \(\hat{\beta_1}\) in (1).

\end{document}
